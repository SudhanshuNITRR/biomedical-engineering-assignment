\documentclass[12pt]{article}

\title{National Institute of Technology,Raipur\\ASSIGNMENT-1 of Biomedical Engineering}
\author{Submitted by: Sudhanshu Patel\\Roll.no: 21111065\\Details: 1st Semester,Biomedical Engineering\\Supervision of : Saurabh Gupta Sir}

\begin{document}
\maketitle
\clearpage
1st. Equipnment Name:  CT SCANNER

\section{INTRODUTION:}
The first commercially available CT scanner was created by British engineer Godfrey Hounsfield of EMI Laboratories in 1972. Computed Tomography (CT) imaging is also known as "CAT scanning" (Computed Axial Tomography). Tomography is derived from the Greek word "tomos" which means "of" or "category" once
"graphia" meaning "to describe".

\section{Components of CT scan machine:\\1.Gantry}
  Gantry computed tomography scanner (CT) is a ring or cylinder, in which a patient is inserted. It is made up of all patient-related equipment, including patient support, a couch, space equipment, and scanner houses.
  
  \section*{2. Detectors:}
   Detectors measure the intensity of radiation transmitted through the patient. CT detector capture the radiation beam from the patient and convert it into electrical signal, which subsequently converted into binary coded information.
   
\section{Type of Detector:}
 1.Gas ionization detector: Convert x-ray energy directly into electrical signal.
 
 
  2.Scintillation detectors: Convert x-ray energy into light. 
  
  3.Data Acquisition System 
  
  \section{IMAGE RECONSTRUCTION:}
   Image reconstruction in CT is a mathematical process that generates tomographic images from X-ray projection data acquired at many different angles around the patient.
   
\section{Conmponent }     
            
\section{WORKING:}
   A CT scan combines a series of X-ray images taken from different angles around our body and uses computer processing to create cross-sectional images (slices) of the bones, blood vessels and soft tissues inside our body. CT scan images provide more-detailed information than plain X-rays do. 
   
\section{USES:}
1) internal and bone injuries from vehicle accidents or other trauma.2)Detect osteoporosis.3)Detect many different types of cancers and determine the extent (spread) of the tumors.4)Determine the cause of chest or abdominal pain, difficulty breathing, and other symptoms.

\section{BENIFITS:}
 1)CT scanning is painless, noninvasive, and accurate.2)A major advantage of CT is its ability to image bone, soft tissue, and blood vessels all at the same time.3)CT exams are fast and simple.4)CT is less sensitive to patient movement than MRI.5)No radiation remains in a patient's body after a CT exam.
 
\section{INDUSTRIAL USE:}
Industrial CT scanning (industrial computed tomography) is a process which utilizes X-ray equipment to produce 3D representations of components both externally and internally. Industrial CT scanning has been utilized in many areas of industry for internal inspection of components. Some of the key uses for CT scanning have been flaw detection, failure analysis, metrology, assembly analysis, image-based finite element methods[60] and reverse engineering applications. CT scanning is also employed in the imaging and conservation of museum artifacts.

\maketitle
\clearpage
2nd.  Equipment name : Pulse Oximeter

\section*{1. INTRODUCTION:}
 The first pulse oximetry was developed in 1972,by Japanese bioengineers, Takuo Aoyagi and Michio Kishi, at a Japanese medical electronic equipment manufacturer, Nihon Kohden, using the ratio of red to infrared light absorption of pulsating components at the measuring site.The pulse oximeter is a small, lightweight device that is used to monitor the amount of oxygen in the body. It is a non-invasive tool that can be attached painlessly to the fingertip. It sends two wavelengths of light through the finger to measure the pulse rate and oxygen in the body.
 

 \section*{2. Component of Pulse Oximeter:}
   The most basic pulse oximeter consists of two LED (one red 660 nm LED and one infrared (IR) 940 nm LED)and a single photodiode (PD) in a reflective or transmissive configuration. The pulse oximeter will pulse the red LED and measure the resulting signal on the PD.
   
\section*{3.  WORKING :}
 Oximeters work by the principles of spectrophotometry: the relative absorption of red (absorbed by deoxygenated blood) and infrared (absorbed by oxygenated blood) light of the systolic component of the absorption waveform correlates to arterial blood oxygen saturations.In pulse oximetry, small beams of light pass through the blood in your finger, measuring the amount of oxygen. According to the British Lung Foundation, pulse oximeters do this by measuring changes in light absorption in oxygenated or deoxygenated blood. This is a painless process. 
 
 \section*{4. USES:}
  Pulse oximetry is a test used to measure the oxygen level (oxygen saturation) of the blood. It is an easy, painless measure of how well oxygen is being sent to parts of our body furthest from your heart.
  
  
  \section*{5. BENIFITS:} 
  1)Pulse Oximeter is a cheap tool.
  
   2)The pulse oximeter is a small, lightweight device.
   
    3)The device is useful in assessing patients with lung disease.
    
    4)Pulse oximetry can also provide feedback about the effectiveness of breathing interventions, such as oxygen therapy and ventilators.
    
\section*{6.  LIMITATION:}
  

 Erroneously low readings may be caused by hypoperfusion of the extremity being used for monitoring (often due to a limb being cold, or from vasoconstriction secondary to the use of vasopressor agents); incorrect sensor application; highly calloused skin; or movement (such as shivering), especially during hypoperfusion. To ensure accuracy, the sensor should return a steady pulse and/or pulse waveform. Pulse oximetry technologies differ in their abilities to provide accurate data during conditions of motion and low perfusion.          
  
 Pulse oximetry also is not a complete measure of circulatory oxygen sufficiency. If there is insufficient bloodflow or insufficient hemoglobin in the blood (anemia), tissues can suffer hypoxia despite high arterial oxygen saturation.
 
 A noninvasive method that allows continuous measurement of the dyshemoglobins is the pulse CO-oximeter, which was built in 2005 by Masimo. By using additional wavelengths, it provides clinicians a way to measure the dyshemoglobins, carboxyhemoglobin, and methemoglobin along with total hemoglobin.
 
 
\maketitle 
\clearpage
3rd. Equipment Name : PH METER

\section*{1. INTRODUCTION:}
  First pH meter was constructed in 1934 by Arnold Beckman. Glass pH electrode that have potential dependent on activity of H+ ions was constructed much earlier, in 1906 by Fritz Haber and Zygmunt Klemensiewicz. It is also called a “potentiometric pH meter” because it measures the variation in electrical potential between a pH electrode and a reference electrode.pH meter,  electric device used to measure hydrogen-ion activity (acidity or alkalinity) in solution. Fundamentally, a pH meter consists of a voltmeter attached to a pH-responsive electrode and a reference (unvarying) electrode.The pH scale is in the range from 1 to 14.Formula for finding pH :pH = -Log10 [H+].

\section*{2. Component of PH METER:}
 A pH meter is consisted of three different parts: 1) an internal electrode, 2) a reference electrode, and 3) a high input impedance meter. Glass probe often contains the two electrodes - internal electrode and reference electrode.
 
 \section*{3. WORKING:}
   The working of pH meter is based on Nernst equation. Nernst equation derives the relation between the electric voltage and ion concentration. The Nernst equation derived for H+ ion concentration is the basis of pH meter. The working principle of pH meter is the potentiometry.
   
\section*{4.  USES :}
A pH meter is an instrument used to measure hydrogen ion activity in solutions - in other words, this instrument measures acidity/alkalinity of a solution. The degree of hydrogen ion activity is ultimately expressed as pH level, which generally ranges from 1 to 14.


\section*{5. BENEFITS:}
1)pH Measurement is inexpensive and robust.

 2)Pocket size pH Meters are user friendly.
 
 3)Readings are accurate and precise.
 
 \section*{6. HISTORY:}
 
 In October 1934, Arnold Orville Beckman registered the first patent for a complete chemical instrument for the measurement of pH, U.S. Patent No. 2,058,761, for his "acidimeter", later renamed the pH meter. Beckman developed the prototype as an assistant professor of chemistry at the California Institute of Technology, when asked to devise a quick and accurate method for measuring the acidity of lemon juice for the California Fruit Growers Exchange (Sunkist).
 
 In the 1940s the electrodes for pH meters were often difficult to make, or unreliable due to brittle glass. Dr. Werner Ingold began to industrialize the production of single-rod measuring cells, a combination of measurement and reference electrode in one construction unit, which led to broader acceptance in a wide range of industries including pharmaceutical production
 
 \section*{7. Maintenance:}
 Because of the sensitivity of the electrodes to contaminants, cleanliness of the probes is essential for accuracy and precision. Probes are generally kept moist when not in use with a medium appropriate for the particular probe, which is typically an aqueous solution available from probe manufacturers. Probe manufacturers provide instructions for cleaning and maintaining their probe designs. For illustration, one maker of laboratory-grade pH gives cleaning instructions for specific contaminants: general cleaning (15-minute soak in a solution of bleach and detergent), salt (hydrochloric acid solution followed by sodium hydroxide and water), grease (detergent or methanol), clogged reference junction (KCl solution), protein deposits (pepsin and HCl, 1% solution), and air bubbles.
 
\maketitle
\clearpage
4th. Equipment Name :  X-RAY MACHINE

\section*{1.  INTRODUCTION:}
  X-rays were discovered in 1895 by Wilhelm Conrad Roentgen (1845-1923) who was a Professor at Wuerzburg University in Germany. Working with a cathode-ray tube in his laboratory, Roentgen observed a fluorescent glow of crystals on a table near his tube. The tube that Roentgen was working with consisted of a glass envelope (bulb) with positive and negative electrodes encapsulated in it. The air in the tube was evacuated, and when a high voltage was applied, the tube produced a fluorescent  glow. Roentgen shielded the tube with heavy black paper, and discovered a green colored fluorescent light generated by a material located a few feet away from the tube.
  
\section*{2.  Component of X-RAY MACHINE:\\1. Operating Console:}
 The operating console allows the radiologic technologist to control the x-ray tube current and voltage so that the useful x- ray beam is of proper quantity and quality. Radiation quantity refers to the number of x-rays or the intensity of the x-ray beam. Radiation quantity is usually expressed in milliroentgens (mR) or milliroentgens/milliampere-second (mR/mAs).The operating console usually provides for control of line compensation, kVp, mA, and exposure time. Meters are provided for monitoring kVp, mA, and exposure time.
 
\section*{2.High Frequency Generator :}
 A high frequency generator powers the X-ray tube. Earlier, high voltage generators were used. High frequency generators are used.
 
 \section*{3.  WORKING:}
  x-ray machines produce a stream of electromagnetic radiation that interacts with an anode in an x-ray tube. ... When x-rays come into contact with our body tissues, they produce an image on a metal film. Soft tissue, such as skin and organs, cannot absorb the high-energy rays, and the beam passes through them.
  
\section*{4.   USES:}
 An X-ray is a common imaging test that's been used for decades. It can help your doctor view the inside of your body without having to make an incision. This can help them diagnose, monitor, and treat many medical conditions. Different types of X-rays are used for different purposes.
 
 
 \section*{5. BENEFITS:}
 noninvasively and painlessly help to diagnose disease and monitor therapy
support medical and surgical treatment planning and
guide medical personnel as they insert catheters, stents, or other devices inside the body, treat tumors, or remove blood clots or other blockages.

\section*{6.Properties :}
X-ray photons carry enough energy to ionize atoms and disrupt molecular bonds. This makes it a type of ionizing radiation, and therefore harmful to living tissue. A very high radiation dose over a short period of time causes radiation sickness, while lower doses can give an increased risk of radiation-induced cancer. In medical imaging, this increased cancer risk is generally greatly outweighed by the benefits of the examination. The ionizing capability of X-rays can be utilized in cancer treatment to kill malignant cells using radiation therapy. It is also used for material characterization using X-ray spectroscopy.

X-rays have much shorter wavelengths than visible light, which makes it possible to probe structures much smaller than can be seen using a normal microscope. This property is used in X-ray microscopy to acquire high-resolution images, and also in X-ray crystallography to determine the positions of atoms in crystals.


\maketitle
\clearpage
5th.  Equipment Name :  CAMERA PILL:

\section*{1.  INTRODUCTION:}
The advancement of our technology today has ead to is effective use and application to the medical fiekl. One effective and purposeful applcation of the advancemen of technology is the process of Endoscopy, which is used to diagnose and examine the conditions of the gastrointestiral tract of the patents. It has been reported that this process is done by inserting an 8mm tube through the mouth, with a camera at one end, and images are shown on nearby monitor, albowing the medics to carefully guide it down to the gullet or stomach However, despite the effectiveness of this process to diagnose the patients, research shows that endoscopy is a pain stacking process not only for the patients, but ako for the doctors and nurses as well From.

\section*{2. Components of CAMERA PILL:}
   The capsule device contains several components including, an external capsule, a transparent optical window, LEDs (light-emitting diodes), a lens, an image sensor, batteries, a radio frequency transmitter, and an antenna. The receiver carried by the patient collects data transmitted from pill-cam.

\section*{3.  WORKING :}
During a capsule endoscopy procedure, you swallow a tiny camera that's about the size of a large vitamin pill. The capsule contains lights to illuminate your digestive system, a camera to take images and an antenna that sends those images to a recorder you wear on a belt.

\section*{4.  USES:}
 The capsule contains lights to illuminate your digestive system, a camera to take images and an antenna that sends those images to a recorder you wear on a belt. Capsule endoscopy is a procedure that uses a tiny wireless camera to take pictures of your digestive tract.
 
 \section*{5.  BENEFITS:}
 The capsule is usually passed out painlessly with the faeces within one to two days. Small Bowel Capsule Endoscopy offers several advantages over traditional endoscopy procedures including it does not require sedation, is less likely to cause discomfort and has fewer potential complications. 
 
 \section*{6.  Manufactures:}
 As of 2015 there were a number of manufacturers. The technology was originally developed by Gabi Iddan and Paul Swain, with the first pill swallowed in 1997; Iddan founded Given Imaging which received FDA approval in 2001.
 
 \section*{7. SIDE EFFECTS: }
 Capsule endoscopy is considered to be a very safe method to determine an unknown cause of a gastrointestinal bleed. The capsule is usually excreted with the feces within 24–48 hours. There has been a report of retention of the capsule for almost four and a half years although the patient was asymptomatic. However, the risk of bowel obstruction may be countered by abdominal X-ray to locate the device for removal by endoscopy or surgery.
 
\section*{8. Procedure:}
As of 2014 research was targeting additional sensing mechanisms and localization and motion control systems to enable new applications for the technology, for example, drug delivery. Wireless energy transmission was also being investigated as a way of providing a continuous energy source for the capsule.
   
 

 
 
 


\end{document}

